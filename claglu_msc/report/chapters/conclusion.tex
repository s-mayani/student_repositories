\chapter{Conclusion}
\label{chapter:conclusion}

In this work we presented an implementation of the Langevin collision operator via a Particle-in-Cell
scheme for modeling the Vlasov-Poisson-Fokker-Planck equation.
We explored multiple methods for solving the electrostatic and Rosenbluth potentials by comparing
their performance to analytical test cases.
Additionally, we demonstrated 2\textsuperscript{nd} order algebraic error convergence for both the dynamical friction
and diffusion term.

Subsequently, we applied the solver to the challenging problem of simulating the disorder induced
heating process (\gls{dih}) in an initially cold sphere.
Analyzing the impact of the coefficients on the normalized emittance, we found the friction
coefficient to be too small to cause any effect.
A possible explanation is that the \gls{dih} problem starts off from
an unphysical zero velocity initial condition.
We showed that the diffusion matrices during the first few plasma periods tend to be negative
definite when close to the simulation boundary. This property inhibits the Euler-Maruyama time
integration method as it makes use of the matrices Cholesky decomposition.
Further investigation showed that the frequency at which such negative definite matrices occur rapidly
decreases after the off-diagonal elements start to show an oscillatory behavior.
This indicates that there is some warm-up time in which assumptions on the collisional operator could
be violated, leading to these unphysical diffusion matrices.

During our investigations of appropriate numerical operators for the Rosenbluth potentials,
we developed a tool that enables the composition of user-defined operators into a single
callable instance that is generated at compile time.
\newpage
\section{Outlook}

The investigation on the Langevin collision operator gave rise to many unexpected and interesting questions which
could be explored in a potential continuation of this project.

We divide this section into suggestions serving two differing purposes:

\begin{enumerate}
    \item Further investigation on the collision operator when applied to the \gls{dih} problem.
    \item Possible algorithmic and general performance improvements.
\end{enumerate}

We suggest a continuation of the investigation on the apparent negative definiteness of the
diffusion matrices in the early stages of the \gls{dih} simulation. Especially, we think is beneficial to
better understand the origin of the noise that impacts the off-diagonal entries. Also, it could help
to inspect how the eigenvalues are distributed between the matrices.
As an intermediate solution, one could use a diagonal restriction of the afflicted matrices, making them
factorizable. The \gls{ldlt} decomposition we currently use would allow this to be implemented without
much overhead. However, we point out that one should be aware of and investigate possible negative
impacts this linearization brings along.

Another direction of research could be, to better understand how the interplay between mesh
spacing and time step influences the collisional coefficients' values. This has partly been explored
by Stoel \cite{stoel} on another test case, though we expect it to be of even higher importance in
the \gls{dih} problem due to the peculiar initial conditions causing a discontinuity at the origin.
We suggest exploring this with methods akin to the one used by Adam
\cite{subcyclingAdam1982}, also known as subcycling, during the early stages of the simulation.
Similarly as the number of macro-particles per cell is low for most of the velocity domain it would be
worthwile investigating how increasing this number reduces the observed error in the collisional
coefficients.

We have shown the correctness of our collisional operator on analytical test cases.
Thus, we suggest testing it on a simpler physical test case to examine if we
encounter similar issues as with the \gls{dih} problem or not.
As an example a stationary initial velocity distribution (i.e. Maxwellian) could be used.
This would also allow investigating the energy dissipation introduced by the Euler-Maruyama integrator.

This report has been a case study of the applicability of a collisional operator to the \gls{dih} problem.
Our main focus did not lie on implementing a highly optimized and efficient solver before we could
show its correctness. Nevertheless, during the development process we have found many ways how one
could optimize it.
The current implementation of the solver supports shared-memory parallelism with \gls{openmp} and
\gls{simd} parallelism via single GPU execution.
Distributed memory parallelism (via \gls{mpi}) is currently not supported due to the domain
decomposition being defined solely by the particle distribution on the spatial grid.
One could implement the concept of super-cells introduced by Qiang \cite{qiang2000self} which uses
separate velocity grids for each \gls{mpi} rank to circumvent these limitations.
Another potential performance improvement concerns the computation of the two collisional terms.
It could be carried out completely asynchronously on the GPU via the CUDA streams concept,
as their computation is independent up to the point where the collisional coefficients are used
to update the particle velocities.
