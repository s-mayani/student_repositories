\chapter{Introduction}
\label{chapter:introduction}

Plasma, often referred to as the fourth state of matter, consists of a gas-like collection of
charged particles (ions and electrons) that exhibit a collective behavior.
Understanding its dynamics under the influence of electric and magnetic fields is of critical
importance for numerous areas in research as well as industry.
A prime example are non-neutral plasmas confined in particle accelerators where particle bunches are exposed to
a range of interesting phenomena which depend on the present plasma state.

In this report, we aim to model a phenomenon arising in ultra-cold plasmas where the transport of
the contained particles is strongly influenced by inter-particle collisions.
Modeling these scattering processes has been an active area of research for many decades
\cite{piwinski1974intra,kubo2001intrabeam}.
The individual scattering events are very weak and often happen over large distances.
Thus, modeling them as binary collisions requires very small timesteps which is not feasible for the
number of particles one usually encounters in dense plasmas.

Instead, researchers have explored methods emanating from plasma kinetic theory which model the scattering
processes via the Fokker-Planck (\gls{fp}) equation \cite{chandrasekhar1943,rosenbluth}.
It can be derived with a Taylor expansion of the statistically averaged effect on the velocity space due to
small-angle deflections. This effect describes advection (via friction) and diffusion in velocity space.
The definition of the friction and diffusion term of the \gls{fp} formalism is not straightforward, thus
researches usually have to state assumptions on the system state in order to approximate them
\cite{manheimer1997langevin,cadjan_ivanov_1999,jonesLangevin_1996}.
In our work we model the collisions via the Langevin formulation of the \gls{fp} equation \cite{cadjan_ivanov_1999}, allowing us to
model them as a stochastic process, providing a velocity dependent
deterministic friction and stochastic diffusion term \cite{stoel}.
This formulation facilitates the use of the Fokker-Planck term as part of the well-established
Particle-in-Cell (\gls{pic}) method \cite{bunemanPIC, dawsonPIC}.

\section{Motivation}

Free Electron Lasers (\gls{fel}) have the ability to generate beams that emit very short coherent light
pulses which exhibit wave lengths down to 0.1\ nm.
This laser is used, among many other applications, to map the atomic structure of proteins with a
high temporal resolution or allows investigating the atomic structure of crystalline matter.
An electron beam is emitted from an electron gun which is subsequently accelerated through an array
of magnets causing a lumping of the beam into bunches.
This process is prone to be negatively impacted by intra-beam scattering of the particles, causing the beam to widen
over time.
As this phenomenon negatively affects the achievable beam brightness, it is important to calibrate an
accelerator to counter this effect as much as possible.

Swiss\gls{fel} is an X-ray \gls{fel} at the Paul Scherrer Institute containing multiple beamlines for producing
many different types of X-ray pulses.
Prat et al. \cite{prat2022energy} have measured in experiments on this machine an energy spread for increased bunch
charges which is a magnitude larger than what their standard simulation codes predicted.
They showed that this blowup is caused mainly by intra-beam scattering and microbunching instabilities.
Being able to model the energy spread correctly is important as it directly influences how much
bunches can be compressed and also defines a lower bound on the achievable wavelength.
An existing method \cite{p3m_ulmer} based on the \gls{p3m} algorithm
(Particle-Particle Particle-Mesh, see Section \ref{section:PIC_method}) has been shown to model
intra-beam collisions adequately, but it does so at a considerable computational cost.
This inhibits simulating bunches over a complete beamline.
The aforementioned method of modeling collisions stochastically via the Langevin formulation promises
better computational complexity, therefore allowing to simulate larger number of particles and simulation domains.
We test the method on an experiment which is governed by disorder induced heating (\gls{dih}) \cite{mitchell2015parallel} and has been
shown to be resolved correctly only if individual particle collisions are considered (via the
\gls{p3m} method \cite{p3m_ulmer}).
This provides us with a challenging test case for the implemented method.

\section{Outline}

We start by introducing the notation used throughout the report, followed by a succinct treatise of the
underlying theory of plasma modeling in the context of kinetic theory, resulting in the Langevin formulation of
the \gls{fp} equation.
Subsequently we discuss the numerical methods used to solve the previously introduced equations and
propose an analytical test case that is used to verify the correctness of our implementation in the following chapter.
We then apply it to the physical test case governed by \gls{dih} and explore what
values the two collisional coefficients attain and how they impact the quantity of normalized emittance.
We conclude this work by highlighting the key findings of our investigation.
Additionally, we propose algorithmic improvements and suggest possible directions for further exploration
of the collisional terms applied in the context of \gls{dih}.
