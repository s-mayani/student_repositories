
\chapter{Introduction}

Many fundamental physics equations governing natural phenomena cannot be solved analytically.
However, using computer simulations we can solve complex equations numerically.
This process can be accelerated using supercomputers.
Supercomputers, or computer clusters, underwent many evolutionary changes
in pursuit of more performance, measured by the number of computations per second,
more precisely, the number of floating point operations per second (FLOP/s).
In the past few years, there was a race to the first exascale supercomputer,
a supercomputer that is capable of calculating at least $10^{18}$ IEEE 754 double precision (64 bit) floating point
operations per second (Exaflop/s).
The term `exascale computing' refers to computing systems being capable of that.
In June of 2022, the Frontier system, located at Oak Ridge National Laboratory (ORNL),
achieved a High-Performance Linpack (HPL) benchmark score of 1.102 Exaflop/s
and was crowned the first true exascale machine \cite{noauthor_june_2023}.

Moving to the era of exascale, heterogeneous computing architectures,
with different kinds of processors and/or possibly different kinds of GPUs in the same system, are unavoidable.
Because of this, we seek to provide tools to computational physicists to carry out
extreme-scale simulations on the next generation of computing architectures.
The IPPL (Independent Parallel Particle Layer) library \cite{frey_ippl-frameworkippl_2023} is a
performance portable framework for Particle-In-Cell (PIC) methods in \texttt{C++}.
It is being developed by collaborators from Paul Scherrer Institute (PSI),
Jülich Supercomputing Center (JSC)
and the University of St Andrews.
Currently, IPPL supports electrostatic PIC simulations.
Development of a full electromagnetic (EM) solver is ongoing.

EM solvers are useful for simulating Free Electron Lasers,
where the radiation interacts with free electrons,
whose motions are governed by the electromagnetic Maxwell equations.

The goal of this thesis is to explore one of the numerical methods used for
EM solvers, the Finite Element Method (FEM) and implement the building blocks
for it. The thesis is structured as follows: First, we present the methodology
and the software framework, then we cover the implementation of the
building blocks and finally we discuss the result with convergence plots, followed by our
conclusion and a few words on future work.
