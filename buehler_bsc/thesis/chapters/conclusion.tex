\chapter{Conclusion}

This thesis and the implemented building blocks for it are the first big step into FEM
for IPPL. It not only contains the building blocks for FEM, it can already be considered a FEM
framework by itself, with the inclusion of the solver.
We have shown the implementation from the ground up and have been able to achieve good
convergence using the solver.

% Design
% Elements
% Finite Element Space
% LagrangeSpace and basis functions
% Quadrature
% Assembly
% Solver

We started by designing the general software architecture of the \texttt{C++} FEM framework in IPPL,
with extensibility in mind.
Then, element classes along with dimension-independent affine transformations were implemented,
followed by the \texttt{FiniteElementSpace} base class handling the mapping of the mesh to the elements.
Then the \texttt{LagrangeSpace} class was implemented, which supports first-order Lagrangian finite elements,
for one- to three-dimensional meshes. It was also designed to be easily extensible to higher-order basis functions.
Next, a base class for numerical quadrature was added which has methods for creating
tensor-product quadrature points and weights for all the reference elements. It serves
as the basis of the Midpoint quadrature rule and the Gauss-Jacobi quadrature rule which were also implemented
in this work.
The latter uses an iterative algorithm to compute the roots of the Jacobi-Polynomial.
With the main building blocks taken care of, a special matrix-free assembly algorithm, interfacing with the CG method,
was implemented for ideal performance on GPUs, in the future.
And finally, the \texttt{FEMPoissonSolver} was added as a proof-of-concept and as an example
for future FEM solvers.

