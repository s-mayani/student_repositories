\chapter{Methodology}

For electromagnetic (EM) simulations, it is common to use the Finite Difference Time Domain (FDTD) scheme \cite{yee_finite-difference_1997}.
The FDTD scheme discretizes the time and space derivatives with central finite differences, hence, it only has 2nd-order accuracy.

Another method that can be used to solve the Maxwell equations is the Finite Element Method (FEM).
When solving EM problems with FEM, the space discretization is done using FEM
and other schemes such as Runge-Kutta methods are used to solve in the time domain.
This is known as Finite Element Time Domain (FETD).

FEM has multiple advantages compared to finite differences. It allows for more complex geometries
and higher-order elements (i.e.\ higher-order basis functions for the finite element spaces) to improve accuracy.
Using higher-order elements to improve the accuracy is known as $p$-refinement.
The higher-order elements allow FEM to achieve better accuracy without having to refine the mesh
and therefore improve the accuracy without really affecting the runtime, scalability and memory footprint.
It is also possible to refine the mesh, just as in the finite difference scheme,
which is known as $h$-refinement. It is even possible to do both with $hp$-refinement.

Furthermore, a matrix-free assembly algorithm can be used, which gives the method an even smaller memory footprint
and thus further improves the performance on GPUs.

In this work, we show an implementation of a Finite Element Method framework in IPPL \cite{frey_ippl-frameworkippl_2023}
with a matrix-free assembly algorithm and the possibility for $p$-refinement.



\section{Finite Element Method (FEM)}


\label{sec:fem_intro}

The Finite Element Method (FEM) is a method to solve partial differential equations (PDEs).
To use it, boundary value problems
need to be formulated into a variational boundary problem (weak form).
After that, the FEM consists of three main steps:
Firstly, discretizing or `meshing' the domain with a finite number of `elements'.
Secondly, approximating the solution of the PDE on each of the elements and finally
assembling the solutions from the elements in a linear system of equations (LSE).
This LSE can then be solved with any LSE solver.
FEM produces sparse problems,
which is why sparse LSE solvers are often used.
In this work, however, we use the conjugate gradient algorithm to solve the LSE instead,
since this allows us to avoid having to explicitly build the left-hand-side matrix (matrix-free).
For more details on the matrix-free method, see section \ref{sec:matrix_free}.

\subsection{Variational formulation}

\begin{definition}[Linear form]
    \label{def:linear_form}
    Given a vector space $V$ over $\R$, a \emph{linear form} $\ell$ is a mapping
    $\ell: V \mapsto \R$ that satisfies
    \cite[Def.~0.3.1.4]{hiptmair_numerical_2023}
    \begin{align}
        \ell(\alpha u + \beta v) = \alpha \ell(u) + \beta \ell(v), \quad
        \forall u,v \in V, \ \forall \alpha, \beta \in \R.
    \end{align}
\end{definition}

\begin{definition}[Bilinear form]
    \label{def:bilinear_form}
    Given a vector space $V$ over $\R$, a \emph{bilinear form} $\mathrm{a}$ on $V$
    is a mapping $\mathrm{a}: V \times V \mapsto \R$, for which it holds
    \cite[Def.~0.3.1.4]{hiptmair_numerical_2023}
    \begin{align}
         & \mathrm{a}\left(\alpha_1 v_1+\beta_1 u_1, \alpha_2 v_2+\beta_2 u_2\right)= \nonumber \\
         & \qquad \alpha_1 \alpha_2 \mathrm{a}\left(v_1, v_2\right)
        +\alpha_1 \beta_2 \mathrm{a}\left(v_1, u_2\right)
        +\beta_1 \alpha_2 \mathrm{a}\left(u_1, v_2\right)
        +\beta_1 \beta_2 \mathrm{a}\left(u_1, u_2\right),                                       \\
        \nonumber                                                                               \\
         & \forall u_i, v_i \in V, \alpha_i, \beta_i \in \mathbb{R}, i=1,2. \nonumber
    \end{align}
\end{definition}

\begin{definition}[Continuous Linear Form]
    \label{def:cont_linear_form}
    Given a normed vector space $V$ with norm $\|\cdot\|$, a linear form $\ell: V \to \R$
    is \emph{continuous} on $V$ if \cite[Def~1.2.3.42]{hiptmair_numerical_2023}
    \begin{align}
        \exists C > 0: \quad |\ell(v)| \leq C\|v\| \quad \forall v \in V,
    \end{align}
    holds for $C \in \R$.
\end{definition}
\begin{definition}[Continuous Bilinear Form]
    \label{def:cont_bilinear_form}
    Given a normed vector space $V$ with norm $\|\cdot\|$,
    a bilinear form $a: V \times V \to \R$ on $V$ is \emph{continuous}, if
    \cite[Def~1.2.3.42]{hiptmair_numerical_2023}
    \begin{align}
        \exists K > 0: \quad |\mathrm{a}(u, v)| \leq K\|u\| \|v\| \quad \forall u,v \in V_0,
    \end{align}
    holds for $K \in \R$.
\end{definition}

\begin{definition}[(Galerkin) Linear Variational Problem (LVP)]
    \label{def:lin_var_prob}
    We define a \emph{linear variational problem (LVP)} as
    \cite[Def.~1.4.1.7]{hiptmair_numerical_2023}
    \begin{align}
        u \in V: \quad \mathrm{a}(u, v) = \ell(v) \quad \forall v \in V, \label{eq:lin_var_prob}
    \end{align}
    where $V$ is a vector space, with norm $\|\cdot\|_V$,
    $\mathrm{a}: V \times V \mapsto \R$ is a continuous bilinear form and
    $\ell: V \mapsto \R$ is a continuous linear form.
    The function $u \in V$ is called the \emph{trial function} and the functions $v \in V$ are called the \emph{test functions}.

    In a generalized linear variational problem, the test and trial functions can be from different vector spaces, defined on the same subspace.
    Then these two vector spaces are called the test space and the trial space. However, in this work, we are only considering
    Galerkin Finite Element Methods where the test and the trial space are the same vector space, by definition.
\end{definition}

\subsubsection{Example: Poisson Equation}
\label{sec:poisson_eq}

The strong form of the Poisson equation with homogeneous Dirichlet boundary conditions is
\begin{align}
    \begin{array}{r l}
        -\Delta u = f & u \in \Omega,          \\
        u        = 0  & u \in \partial \Omega,
    \end{array}
    \label{eq:poisson_strong_form}
\end{align}
where $\Omega$ is the domain, $\partial \Omega$ is the boundary of the domain $\Omega$ and $f: \Omega \mapsto \R$ is the source function.

The weak-form or variational equation of the Poisson equation with homogeneous Dirichlet boundary conditions is
\begin{align}
    \int_\Omega \nabla u \cdot \nabla v \ d\vec{x} = \int_\Omega f v \ d\vec{x},
    \label{eq:poisson_weak_form}
\end{align}
where $v \in V$ is the test function and $u \in V$ is the trial function,
and $f: \Omega \to \R$ is the source function.
The derivation can be found in \cite{hiptmair_numerical_2023}.

The bilinear form and the linear of this problem are therefore
\begin{align}
    \mathrm{a}(u, v) & = \int_\Omega \nabla u \cdot \nabla v \ d\vec{x}, \\
    \ell(v)          & = \int_\Omega f v \ d\vec{x}.
\end{align}

\subsection{Discretization (Meshing)}

Recall the definition of the linear variational problem (Def. \ref{def:lin_var_prob}):

\begin{align}
    u \in V: \quad \mathrm{a}(u, v) = \ell(v) \quad \forall v \in V, \tag{\ref{eq:lin_var_prob}}
\end{align}

where $\mathrm{a}: V \times V \mapsto \R$ is the bilinear form and $\ell: V \mapsto \R$
is the linear form, $u$ is the trial function and $v$ are the test functions.

The first part of the discretization (meshing) step is the replacement of the
infinite-dimensional vector space $V$ in the linear variational problem with a
finite-dimensional subspace $V_{h} \subset V$ \cite[Chapter~2.2.1]{hiptmair_numerical_2023}.

\begin{definition}[Discrete (linear) variational problem (DVP)]
    \label{def:discrete_var_prob}
    The \emph{discrete variational problem (DVP)} is defined as
    \cite[Def.~2.2.1.1]{hiptmair_numerical_2023}
    \begin{align}
        \label{eq:discrete_var_eq}
        u_h \in V_{h}: \quad \mathrm{a}(u_h, v_h) = \ell(v_h), \quad \forall v_h \in V_{h},
    \end{align}
    where $u_h$ is the discretized trial function or \emph{Galerkin solution},
    $v_h$ are the discretized test functions, $\mathrm{a}: V_h \times V_h \to \R$
    is a continuous bilinear form and $\ell: V_h \to \R$ is a continuous linear form.
\end{definition}

The next step in the Galerkin Discretization is the definition of the basis
functions for the discrete variational problem.

We choose an ordered basis $\{b_h^1, \dots, b_h^N\}$ of $V_{h}$ with $N := \dim V_{h}$
where for every $v_h \in V_h$ there are unique coefficients $\nu_i \in \R, i \in \{ 1, \dots, N \}$,
such that $v_h = \sum_{i=1}^{N} \nu_i b_h^i$ \cite[Def.~0.3.1.2]{hiptmair_numerical_2023}.

Inserting this basis representation into the variaional equation \ref{eq:discrete_var_eq} yields:
\begin{align}
     & v_h \in V_{h} \Rightarrow v_h = \nu_1 b_h^1+\cdots+\nu_N b_h^N, \quad \nu_i \in \mathbb{R}, \\
     & u_h \in V_{h} \Rightarrow u_h = \mu_1 b_h^1+\cdots+\mu_N b_h^N, \quad \mu_i \in \mathbb{R},
\end{align}

where the number $N$ is the dimension of the discrete vector space $V_{h}$ and $\nu_i, \mu_i \in \R, \ i \in \{1, \dots, N\}$
are unique coefficients.

The basis functions also have to satisfy the cardinal basis property, as given by definition \ref{def:cardinal_basis_property}.

\begin{definition}[Cardinal basis property]
    \label{def:cardinal_basis_property}

    \begin{align}
        b_h^j(\vec{x}_i) = \begin{cases}
                               1 & i = j,      \\
                               0 & \text{else}
                           \end{cases}, \quad
        i,j \in \{ 1, ..., N \}.
    \end{align}
\end{definition}

Note: In this work, we refer to the basis functions on the global vector space as the
basis functions and to the basis functions restricted to an element $K$ as the
\emph{shape functions}.

\begin{definition}[Shape function]
    For an element $K$, we define the \emph{shape functions} as the basis functions,
    restricted to the element $K$:
    \begin{align}
        b^i_K := b^i_{h|K}.
    \end{align}
\end{definition}
Note: The letter $K$ for the shape function $b^i_K$ may be omitted where the context is clear.

Inserting the definitions of the basis functions into the variational equation %\cite[Chapter~2.2.2]{hiptmair_numerical_2023}
\begin{align}
    \mathrm{a}(u_h, v_h)                                                                                            & = \ell(v_h) \quad                                 & \forall u_h, v_h \in V_{0,h},                                   \\
    \sum_{k=1}^N \sum_{j=1}^N \mu_k \nu_j \, \mathrm{a}\left(b_h^k, b_h^j\right)                                    & = \sum_{j=1}^N \nu_j \ell\left(b_h^j\right) \quad & \forall \nu_1, \dots, \nu_N, \mu_1, \dots \mu_N \in \mathbb{R}, \\
    \sum_{j=1}^N \nu_j\left(\sum_{k=1}^N \mu_k \, \mathrm{a}\left(b_h^k, b_h^j\right)-\ell\left(b_h^j\right)\right) & = 0 \quad                                         & \forall \nu_1, \dots, \nu_N, \mu_1, \dots \mu_N \in \mathbb{R}, \\
    \sum_{k=1}^N \mu_k \, \mathrm{a}\left(b_h^k, b_h^j\right)                                                       & = \ell\left(b_h^j\right)                          & \text { for } j=1, \dots, N.
\end{align}

\begin{definition}[Stiffness matrix]
    \label{def:stiffness_mat}
    The stiffness matrix (or Galerkin matrix) is the matrix of the bilinear form evaluations
    defined as:
    \begin{align}
        \mat{A} = \left[ \mathrm{a}(b_h^j, b_h^i) \right]^N_{i,j=1}, \ \in \R^{N,N}.
        \label{eq:stiffness_mat}
    \end{align}
\end{definition}

\begin{definition}[Load vector]
    \label{def:load_vec}
    The load vector (or right-hand-side vector) is the vector of linear form evaluations
    defined as:
    \begin{align}
        \vec{\varphi} = \left[ \ell(b_h^i) \right]^N_{i=1} \in \R^{N}.
        \label{eq:load_vec}
    \end{align}
\end{definition}

To compute each entry of the stiffness matrix and load vector,
the integrals from the bilinear form and linear form are approximated.
This is done with numerical quadrature (numerical integration).

We arrive at the linear system of equations (LSE):

\begin{align}
    \mat{A}\vec{\mu} = \vec{\varphi}.
\end{align}




\section{Matrix-free Method for FEM computations}

\label{sec:matrix_free}

In 2017, Ljungkvist showed that matrix-free finite element algorithms
have many benefits on modern manycore processors and graphics cards (GPUs)
compared to alternative sparse matrix-vector products \cite{ljungkvist_matrix-free_2017}.
Additionally, in 2023, Settgast et.\ al showed that in the context of the
conjugate gradient (CG) method, the matrix-free approach compares favorably
even for low-order FEM \cite{settgast_performant_2023}.

Motivated by those findings, we chose to implement a matrix-free assembly algorithm that integrates
into the conjugate gradient algorithm, which is already implemented in IPPL.

The CG method is an iterative method and terminates once the norm of the residual is smaller than the specified tolerance $\epsilon \in \R_{>0}$.
The pseudocode for the CG algorithm in IPPL is given in Algorithm \ref{alg:conjugate_gradient}.


\begin{algorithm}[h]

    %\vspace{0.3cm}
    $\vec{x} \gets \text{initial guess, }(\text{usually }\vec{0})$

    $\vec{b} \gets \vec{\varphi}$

    $\vec{p} \gets \mat{A} \vec{x}$

    $\vec{r} \gets \vec{b} - \vec{p}$

    \While{$\|\vec{r}\|_2 < \epsilon$}{
        $\vec{z} \gets \mat{A}\vec{p}$

        \vspace{0.2cm}
        $\displaystyle \alpha \gets \frac{\vec{r}^\top \vec{r}}{\vec{p}^\top \vec{z}}$
        \vspace{0.2cm}

        $\vec{x} \gets \vec{x} + \alpha \vec{p}$

        $\vec{r}_\text{old} \gets \vec{r}$

        $\vec{r} \gets \vec{r} - \alpha \vec{z}$

        \vspace{0.2cm}
        $\displaystyle \beta \gets \frac{\vec{r}^\top \vec{r}}{\vec{r}_\text{old}^\top \vec{r}_\text{old}}$
        \vspace{0.2cm}

        $\vec{p} \gets \vec{r} + \beta \vec{p}$
    }
    \vspace{0.3cm}
    \caption{Pseudocode of the CG algorithm implemented in IPPL.}
    \label{alg:conjugate_gradient}
\end{algorithm}

In the CG method, the stiffness matrix $\mat{A}$ is used in the initialization and in the while loop itself, and in both cases, it is multiplied with a vector.
The implemented FEM framework provides an assembly function to replace this matrix-vector product $\mat{A}\vec{x}$.
The assembly function will return the same resulting vector, but without building the full stiffness matrix $\mat{A}$, thus being ``matrix-free''.

More detailed information regarding the implementation of this assembly algorithm is given in section \ref{sec:assembly}.



\section{Independent Parallel Particle Layer (IPPL)}

The Independent Parallel Particle Layer (IPPL) \cite{frey_ippl-frameworkippl_2023} \cite{muralikrishnan_scaling_2022}
is a performance portable \texttt{C++} library for Particle-Mesh methods.
It is a portable, massively parallel toolkit using the Message Passing Interface (MPI) for inter-processor communication,
HeFFTe \cite{ayala_heffte_2020} as a Fast Fourier Transform (FFT) library and Kokkos \cite{carter_edwards_kokkos_2014} for hardware portability.

\section{Arguments against using an external FEM library}

The Finite Element Method has been studied in great detail already and has been implemented for many languages, environments and use cases.
There are also many \texttt{C++} FEM libraries, for example, MFEM \cite{anderson_mfem_2021} and deal.II \cite{bangerth_dealiigeneral-purpose_2007}.

The possibility of interfacing IPPL with an external library to introduce the Finite Element Method was considered,
but it was decided against for several reasons.
Firstly, new dependencies can bring problems in the future and they further complicate the compilation and installation of IPPL.
Secondly, almost all external libraries use non-standard or custom datatypes.
Since the performance of IPPL on supercomputers is very important, it is crucial to avoid data copies,
which would occur when using a different data type to support an external library,
as they incur high data movement costs, especially on GPUs.

\section{Focus and Scope of this Thesis}

The focus of this Bachelor thesis is the software design and implementation
of the building blocks for the Finite Element Method in the IPPL \cite{frey_ippl-frameworkippl_2023} library.
The focus was on structured, rectilinear grids with rectangular hexahedral (``brick'') elements and the possibility for $p$-refinement.

The term `building blocks' refers to the different parts that are necessary for a functioning FEM implementation.
For example, a class for numerical integration (quadrature),
classes for mesh and degree of freedom (DOF) index mapping and assembly functions for building the resulting linear system of equations.

The framework implemented in this thesis supports first-order Lagrangian finite elements in
one to three dimensions with a matrix-free assembly function that interfaces directly into the CG algorithm.
It supports the basic midpoint quadrature rule and the polynomial Gauss-Jacobi quadrature rule.
For essential boundary conditions, only homogeneous Dirichlet boundary conditions are supported at the moment.

Additionally, a solver to solve the Poisson equation was implemented as a proof-of-concept.
