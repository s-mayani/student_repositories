\chapter{Conclusion}
%\begin{itemize}
%    \item We where able to implement the Nédélec basis functions for IPPL.
%    \item Future work needed. Namely better incorporation into the PIC loop, more domain boundaries, higher order,...
%    \item FEMVector paved way for implementation of more things, higher order and new spaces.
%\end{itemize}

The initial goal that we had was to implement the Nédélec space in \texttt{IPPL}, based on the method we laid out in Section \ref{sec:method} and the experimental results we presented in Section \ref{sec:experiments} we are rather confident that we succeeded in our initial goal, but there are still more things that could be done. For example right now we only support zero Dirichlet boundary conditions, here some more work can be done to expand to more types of boundaries such as non-zero Dirichlet or periodic, all this would entail expanding the functionalities of the \texttt{FEMVector}. Also the Nédélec space needs to be better integrated into the PIC loop, for both how information is passed from PIC to FEM and the other way around. Here some uniform interface could be designed, which is independent of the finite element space used. For the passing of information from PIC to FEM, the function values could be provided at the quadrature nodes such that no additional interpolation has to be done, this would lead to a dependence on the quadrature rule being used by the FEM space, but this seems like much less of a problem compared to a dependence on the finite element space and the DOFs within it. For the exchange of information from FEM to PIC we would say that our current setup of being able to evaluate the solution function at arbitrary points is already pretty good, as we could have that PIC provides the location of all the particles and FEM then simply returns the solution for each of those particles. One could also think about implementing higher order Nédélec spaces. One other interesting thing that should be looked into is that we are not exchanging values over the corners in the halo communication, as seen in Figure \ref{fig:domain_decomp2d}. In our test we did not see any difference in the resulting error between cases where the domain was decomposed in a way where subdomain corners exists where such corner values would have to be exchanged, and decompositions where we do not have such subdomain corners. It also is quite interesting that this phenomenon exists for both 2D and 3D, but here we have that the singular corner values which we do not exchange for the 2D case become a ``line'' for the 3D case where we have to exchange them, so the number of values go from \(\Theta(1)\) to \(\Theta(n)\). This leads us to believe that a reason behind why the values do not need to be exchange might be because there are only constant many of them and we are only missing a halo exchange therefore they are somewhat close to correct and then CG is able to take care of this, but this only makes limited sense, as FEM generally is sensitive to error in a singular DOF and we should loose the second order convergence. We are therefore not entirely sure what the reason behind this is and therefore more research in this area would be required. Based on all this we see that there is still more work that can be done, but we were able to lay the foundation of the Nédélec space.\medskip

While our implementation of the Nédélec space laid the foundation for more possibilities regarding it, we also have that the \texttt{FEMVector} laid the foundation for more development itself. Before we were limited to DOFs on the vertices of the mesh, which meant we only could use the first order Lagrange space, but with the development of the \texttt{FEMVector} we are now able to have DOFs at arbitrary points of the mesh, which makes it possible to implement higher order spaces and entirely new ones. For example the Lagrange space can now be expanded from only supporting first order to supporting second or third order. So while the \texttt{FEMVector} might not be the flashiest new thing we still think that it can play an important role in the future development of FEM inside of \texttt{IPPL}.